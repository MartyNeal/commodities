\documentclass[10pt]{article}
\usepackage{fullpage, graphicx, url}
\usepackage[left=2cm,top=1cm,right=2.5cm,bottom=1cm,nohead,nofoot]{geometry}
\setlength{\parskip}{5mm plus5mm minus2mm}
\setlength{\parindent}{0mm}

\title{Machine learning optimal parameters for a channel breakout system for trading
commodities}
\author{Martin Neal\\
  Becky Engley}
\date{\today}


\newenvironment{packed-item}{
  \begin{itemize}
    \setlength{\itemsep}{1pt}
    \setlength{\parskip}{0pt}
    \setlength{\parsep}{0pt}
}{\end{itemize}}

\begin{document}
\maketitle

\section{Dataset}

We have financial market data for eighteen commodities over a period of thirty
years.  The commodities include live cattle, pork bellies, corn, wheat, lean
hogs, crude oil, sugar, unleaded gasoline, heating oil, soybeans, cotton, orange
juice, and soybean oil.  Some commodities are traded during multiple seasons.
The data includes open, high, low, and closing values for each commodity on
every trading day. The data was obtained
from:

http://www.normanshistoricaldata.com

\section{Background}

A seasonal trading system leverages the normal fluctuations in supply and demand for
certain commodities that occur every year in fairly consistent patterns.  A
commodity is either traded short, long, or bi-directionally.  If a trader expects
an uptrend in the value of a commodity, he will trade long, buying before
selling.  If a trader expects a downtrend, he will trade short, selling before
buying.

In his book, Superstar Seasonals, John Momson defines a channel breakout system
as a model for predicting uptrends and downtrends.  According to his system, a
commodity may be traded between a range of dates specified by the parameters
$entry\_window\_open$ and $entry\_window\_close$.  A trade must be completed no
later than the $exit\_trade$ date, a third parameter of the system.

A channel is defined by an upper channel line and a lower channel line.  The
upper channel line is determined by the maximum $high$ value over the previous
$m$ days.  The lower channel line is determined by the minimum $low$ value
over the previous $n$ days.  Here, $m$ and $n$ are also parameters of the
system.  If trading long, the upper channel line is the $entry$ line and the lower
channel line is the $trail$-$stop$ line.  If trading short, the channel lines are
reversed.

The sixth and final system parameter, the $stop$-$loss$ threshold, serves as a
safety net to prevent large losses.  When trading short, the stop-loss is
defined by the maximum $high$ value over the previous $q$ days, where $q$ is
strictly less than $m$.  When trading long, the stop-loss is defined by the
minimum $low$ value over the previous $q$ days, where $q$ is strictly less than
$n$.  The trail-stop line protects profits, while the stop-loss threshold
minimizes losses.

According to the system, if the current value of the commodity crosses the entry
line, it indicates the beginning of a new trend and the trade is begun.  When
the current value of the commodity crosses the trail-stop line, the trade is
completed.

\section{Project idea}

We intend to machine learn optimal values for the six trading system parameters.
Optimal values maximize the profit earned as a result of trading the system.  We
will approach this optimization problem using two different machine learning
techniques: simulated annealing, and genetic algorithms.

\subsection{Simulated Annealing}

Simulated Annealing combines the best of hill climbing and random walk heuristic
algorithms.  The major problem with hill climbing algorithms is that they can
get stuck on local maxima, because they never move downhill.
Random walks, however, are guaranteed to find the global maximum but take far
too long to do so.  Simulated Annealing combines these approaches, yielding both
efficiency and completeness.  We present the psuedo-code for the Simulated
Annealing algorithm below.

\setlength{\parindent}{5mm}
\indent simulated-annealing()\\
\indent \indent current $\leftarrow$ 6 random parameter values\\
\indent \indent for t $\leftarrow$ 1 to $\infty$\\
\indent \indent \indent $\eta \leftarrow f$(t)\\
\indent \indent \indent next $\leftarrow$ successor(current)\\
\indent \indent \indent $\Delta E \leftarrow$ value(next) - value(current)\\
\indent \indent \indent if($\Delta E > 0$)\\
\indent \indent \indent \indent current $\leftarrow$ next\\
\indent \indent \indent else\\
\indent \indent \indent \indent current $\leftarrow$ next only with probability $e^{\Delta E\eta}$\\
\indent \indent return current\\
\setlength{\parindent}{0mm}

The value function trades the system with the six parameters and returns the
profit made (or lost).  current and next are both nodes.  In this context, a
node is a set of values for the six parameters.  $\eta$ controls the probability
of downward steps.  $f$(t) is a decreasing function of time; it decreases the
probability of downward steps as time increases.  This may be done in any number
of ways (e.g. linearly, exponentially).  The choice for the successor function
greatly affects the learning speed of the algorithm.  Our successor function
will adjust exactly one of the six parameters in current by a random amount,
$\delta$, which may be either uniformly or normally distributed.  As a variation
on this function, we can change multiple parameters at once.  We intend to
experiment with all of these variations.

\subsection{Genetic Algorithm}

Genetic Algorithms model evolutionary processes.  In our proposed
implementation, a set of six system parameters represents an individual.  Many
individuals form a population which is repeatedly bred and then culled.
Breeding swaps random parameters from two or more parents to create children.
Culling evaluates each individual using a fitness function, and eliminates unfit
individuals from the population.  The fitness function returns the profit earned
by trading the channel breakout system using the individuals' six parameters.
We present the psuedo-code for the Genetic Algorithm below.

\setlength{\parindent}{5mm}
\indent genetic-algorithm(population)\\
\indent \indent do\\
\indent \indent \indent for i $\leftarrow$ 1 to size(population)\\
\indent \indent \indent \indent parents $\leftarrow$ random-subset(population,size)\\
\indent \indent \indent \indent children $\leftarrow$ reproduce(parents)\\
\indent \indent \indent \indent children $\leftarrow$ mutate(children) with small random probability\\
\indent \indent \indent \indent population.add(children)\\
\indent \indent \indent population $\leftarrow$ cull(population,threshold)\\
\indent \indent until (enough time has elapsed)\\
\indent \indent return best-individual(population)\\
\setlength{\parindent}{0mm}

This generic algorithm leaves a lot to be decided by the implementer.
The initial population size may vary.  The mutate probability may be changed.
The mutation of an attribute may be uniformly random or normally distributed.
The reproduce function may take between two and six parents.  These parents
randomly swap attributes to produce one or more children, which are added to the
population.  The population is then culled, which preserves a threshold
number of individuals, removing all others as unfit.

\section{Required software and midway report goals}
To implement our project idea, we will need to write a series of programs.  One
program will trade the channel break-out system using the six parameters and
return the profit earned.  The other two programs will implement the simulated
annealing and genetic algorithms as described above.  Our midway report goals
are to have the trading system implemented along with at least one of the
algorithms.  We would also like to have some provisional data collected by then.

\bibliographystyle{plain}
\bibliography{571writeup}
\nocite{*}
\end{document}

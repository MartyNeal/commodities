\documentclass[12pt]{article}
\usepackage{graphicx, url}
\usepackage{listings,color}
\usepackage[left=2cm,top=1.5cm,right=2.5cm,bottom=1cm]{geometry}
\setlength{\parskip}{5mm plus5mm minus2mm}
\setlength{\parindent}{0mm}


\begin{document}

\title{Machine learning optimal parameters for a channel breakout system for trading
commodities}
\author{Martin Neal\\
  Becky Engley}
\date{\today}
\maketitle

\section{Introduction}

The commodities market is an exchange where raw materials and agricultural
products are traded. Commodities are usually traded via futures contracts. A
farmer can sell a futures contract on his crop, months or even years in advance
of the harvest, guaranteeing the price he will recieve at harvest time. A
grocery chain can buy the contract, with the confidence that the price will not
go up when the farmer delivers. Futures contracts protect farmers and buyers
from unexpected price changes.

Investors can also buy and sell the futures contracts, attempting to capitalize
on increases and decreases in price. If an investor expects an increase in the
value of a commodity, he will buy a contract while the price is low, and then
sell his contract at a later time, when the price is higher. In addition to
buying before selling, it is also possible to sell a contract before buying it
back, which an investor may do if he expects a commoditiy's value to decrease.
Buying before selling is referred to as \emph{trading long}; selling before
buying is known as \emph{trading short}.

It is often difficult for investors to determine when to buy and when to sell.
For example, it is impossible to recognize a minimum in price, until after the
price has increased above the minimum. Investors are left guessing as to
whether the price will continue to increase, or immediately begin decreasing
again. Guessing often proves to be unprofitable.

Momsen describes an automated trading system that attempts to predict trends in
commodity prices, providing the investor with a guide to buying and selling.
Momsen defines a pair of channel lines above and below the commodity's price.
The lines are defined using the price history. When the commodity's value
undergoes a change in trend, it crosses one of the channel lines, alerting the
investor to the change. At this point, the investor can decide to buy or
sell, appropriately.

Momsen's channel breakout system relies on a set of six parameters to define
the channel lines and date ranges between which an investor may buy and sell.
This paper aims to data mine commodity market prices, in order to learn better
parameters for Momsen's system. We use two machine learning algorithms to
search the parameter space; simulated annealing, and a genetic algorithm.

\vspace{-15pt}

\section{Related Work}

\vspace{-5pt}

Momsen's book, Superstar Seasonals, describes his channel breakout system as a
model for predicting uptrends and downtrends.  According to his system, a
commodity may be traded between a range of dates specified by the static
parameters $entry\_window\_open$ and $entry\_window\_close$.  A trade must be
completed no later than the $exit\_trade$ date, a third parameter of the
system.

Momsen defines an upper channel line and a lower channel line.  The upper
channel line is determined by the maximum high value over the previous m days.
The lower channel line is determined by the minimum low value over the previous
n days.  Here, m and n are also parameters of the system and are different for
each commodity.  If trading long, the upper channel line is the entry line, and
m is referred to as the $entry\_threshold$. The lower channel line is the
trail-stop line, and n is referred to as the $trail\_stop\_threshold$.  If
trading short, the channel lines are reversed.

The sixth and final system parameter, the $stop\_loss\_threshold$, serves as a
safety net to prevent large losses.  When trading short, the stop-loss is
defined by the maximum high value over the previous q days, where q is
strictly less than m.  When trading long, the stop-loss is defined by the
minimum low value over the previous q days, where q is strictly less than
n.  The trail-stop line protects profits, while the stop-loss line minimizes
losses.

According to the system, if the current value of the commodity crosses the entry
line, it indicates the beginning of a new trend, and the trade is begun.  When
the current value of the commodity crosses the trail-stop line, the trade is
completed.  More often than not, the cross of a channel line incorrectly
predicts the beginning of a new trend.  In these cases, small losses are incurred.
However, when the system correctly predicts a long trend, the profits generated
far outweigh small losses.

Over the past thirty years, Momsen has traded this system, with outstanding
results.  We intend to extend his work by machine learning optimal values for
the six trading parameters.

\section{Methods}

Local search algorithms are used to maximize an objective function. Our
objective function is our algorithm for the channel breakout system, which
takes the six trading parameters as input, and returns the profit
earned. Optimal parameters maximize the profit earned as a result of trading the
system. We first discuss the algorithm for our objective function. Then, we
present the search algorithms that we use to explore the parameter space.

\subsection{Trading the System}

The channel breakout system processes financial market data for an individual
commodity, over a single contract year.  We describe our algorithm for the
channel breakout system below, along with our data set.

\vspace{25pt}
\subsubsection{Data Set}

We have financial market data for fourteen commodities over a period of thirty
years.  Because the learning system is evaluated per contract year, each year
comprises a single data point.  We have approximately thirty data points for
each commodity.  The commodities include live cattle, pork bellies, corn,
wheat, lean hogs, crude oil, unleaded gasoline, heating oil, and orange juice.
Some commodities are traded during multiple seasons.  The data includes open,
high, low, and closing values for each commodity for every trading day.

The data set is partitioned into two sets. For each commodity, we reserve one
third of the data points for testing; the remaining two thirds are used to train
the system. We use the earlier years of data for training, and the more recent
years for testing. We do not reserve any data for validation, as we do not
intend to validate our predetermined model.

\subsubsection{The Channel Breakout System}

The algorithm for the channel breakout system takes in the six trading
parameters, and returns the total profit earned. Our $TradeSystem()$ function
is our objective function, the return value of which we are trying to maximize.
It is used as the $value()$ function for the Simulated Annealing algorithm, and
the $fitness()$ function for the Genetic algorithm, both defined in a later
section.
\vspace{25pt}

\setlength{\parindent}{5mm}
\indent TradeSystem(entry\_window\_open, entry\_window\_close, exit\_date,\\
\indent \indent \indent \indent \indent entry\_threshold, trail\_stop\_threshold, stop\_loss\_threshold)\\\\
\indent \indent compute entry, trail\_stop and stop\_loss channels, using thresholds\\\\
\indent \indent while(entry\_window\_open $<$ current\_date $<$ entry\_window\_close OR in\_trade)\\
\indent \indent \indent if(not in\_trade)\\
\indent \indent \indent \indent if(entry\_channel is crossed)\\
\indent \indent \indent \indent \indent in\_trade $\leftarrow$ TRUE\\
\indent \indent \indent \indent \indent entry\_price $\leftarrow$ current\_price\\
\indent \indent \indent \indent \indent stop\_loss\_value $\leftarrow$ stop\_loss\_channel[current\_date]\\
\indent \indent \indent else\\
\indent \indent \indent \indent if(trail\_stop\_channel is crossed OR stop\_loss\_value is crossed OR\\
\indent \indent \indent \indent \indent current\_date = exit\_date)\\\\
\indent \indent \indent \indent \indent in\_trade $\leftarrow$ FALSE\\
\indent \indent \indent \indent \indent exit\_price $\leftarrow$ current\_price\\
\indent \indent \indent current\_date $\leftarrow$ current\_date + 1\\\\
\indent \indent compute profit using entry\_price and exit\_price\\
\indent \indent return profit\\
\setlength{\parindent}{0mm}

\pagebreak
We may repeatedly enter and then exit a trade many times over the course of one
trading year. When in a trade, we check every day, to determine if we should
exit.  When not currently in a trade, we check to determine if we should enter,
until the entry window closes. When we enter, we compute the entry price based
on the value at which we crossed the entry channel line. The stop-loss value is
also based on this cross-point. When we exit, we compute the exit price based on
the value at which we crossed the closer of the trail-stop channel and the
stop-loss value. The profit for this trade is the difference between the entry
and exit prices.

\subsection{Searching the Parameter Space}

Optimal parameters maximize the profit earned as a result of trading the
system.  We approach this optimization problem using two different machine
learning techniques: simulated annealing, and genetic algorithms.  Below we
present these two algorithms, along with an algorithm for a random learner,
which we use as a baseline comparison.

\subsubsection{Simulated Annealing}

Simulated Annealing combines the best of hill climbing and random walk
heuristic algorithms.  Hill climbing algorithms learn quickly, but they
generally only find local maxima, because they never move downhill. Random
walks are guaranteed to find the global maximum but take far too long to do so.
Simulated Annealing combines these approaches, yielding both efficiency and
completeness.  We present the psuedo-code for the Simulated Annealing algorithm
below.

\vspace{25pt}
\setlength{\parindent}{5mm}
\indent SimulatedAnnealing(number\_iterations)\\\\
\indent \indent current $\leftarrow$ 6 random parameter values\\
\indent \indent max $\leftarrow$ current\\\\
\indent \indent for t $\leftarrow$ 1 to number\_iterations\\\\
\indent \indent \indent next $\leftarrow$ successor(current, n, dist\_type)\\
\indent \indent \indent $\Delta E \leftarrow$ value(next) - value(current)\\\\
\indent \indent \indent if($\Delta E > 0$)\\
\indent \indent \indent \indent current $\leftarrow$ next\\
\indent \indent \indent else\\
\indent \indent \indent \indent current $\leftarrow$ next, only with probability $P(\Delta E) \cdot f(t)$\\\\
\indent \indent \indent if(value(current) $>$ value(max))\\
\indent \indent \indent \indent max $\leftarrow$ current\\\\
\indent \indent return max\\
\setlength{\parindent}{0mm}

\pagebreak
The $value()$ function trades the system with the six parameters and returns the
profit made (or lost).  Current and next are both nodes.  In this context, a
node is a set of values for the six parameters. $f(t)$ is a linearly decreasing
function of time; it decreases the probability of downward steps as time
increases. $P(\Delta E)$ is given by the p-value for $\Delta E$, from a normal
distribution.  We use previous values of $\Delta E$ to compute the mean and
standard deviation for the normal distribution.

The choice of $successor()$ function greatly affects the learning speed of the
algorithm.  Our $successor()$ function adjusts $n$ of the six parameters in
current by a random amount, $\delta$, which may be sampled from one of three
distributions: a uniformly random distribution, a normal distribution, or a
constant distribution (i.e., $\delta$=1).  The parameter $dist\_type$ specifies
the distribution to use.


\subsubsection{Genetic Algorithm}

Genetic Algorithms model evolutionary processes.  In our proposed
implementation, a set of six system parameters represents an individual.  Many
individuals form a population which is repeatedly bred and then culled.
Breeding swaps random parameters from two or more parents to create children.
Culling evaluates each individual using a $fitness()$ function, and eliminates
unfit individuals from the population.  The $fitness()$ function returns the
profit earned by trading the channel breakout system using the individual's six
parameters.  We present the psuedo-code for the Genetic Algorithm below.

\vspace{10pt}
\setlength{\parindent}{5mm}
\indent Genetic(number\_iterations)\\\\
\indent\indent population $\leftarrow$ createPopulation(size)\\\\
\indent \indent for t $\leftarrow$ 1 to number\_iterations\\
\indent \indent \indent for i $\leftarrow$ 1 to size(population)\\\\
\indent \indent \indent \indent parents $\leftarrow$ randomSubset(population, size)\\
\indent \indent \indent \indent children $\leftarrow$ reproduce(parents)\\
\indent \indent \indent \indent children $\leftarrow$ mutate(children, n, dist\_type) with small random probability\\
\indent \indent \indent \indent population $\leftarrow$ add(population, children)\\\\
\indent \indent \indent population $\leftarrow$ cull(population, threshold)\\\\
\indent \indent return bestIndividual(population)\\
\setlength{\parindent}{0mm}

This generic algorithm leaves a lot of room for experimentation. The initial
population size may vary.  The probability of mutation may be changed.  The
mutation of an attribute may be uniformly or normally distributed.  The
$reproduce()$ function may take between two and six parents.  These parents
randomly swap attributes to produce one or more children, which are added to
the population.  The population is then culled, which preserves a threshold
number of individuals, removing all others as unfit. We intend to experiment
with many of these variations.

\subsubsection{Random Walk}

Here we describe our random walk algorithm.  The algorithm is similar to
simulated annealing, in that the next node is chosen via a $successor()$
function (where again, a node is a set of six parameters). Unlike simulated
annealing, our random walk algorithm has no preference for climbing uphill. We
present the psuedo-code for the algorithm below.

\vspace{10pt}
\setlength{\parindent}{5mm}
\indent RandomWalk(number\_iterations)\\\\
\indent \indent current $\leftarrow$ 6 random parameter values\\
\indent \indent max $\leftarrow$ current\\\\
\indent \indent for t $\leftarrow$ 1 to number\_iterations\\
\indent \indent \indent current $\leftarrow$ successor(current, n, dist\_type)\\\\
\indent \indent \indent if(value(current) $>$ value(max))\\
\indent \indent \indent \indent max $\leftarrow$ current\\\\
\indent \indent return max\\
\setlength{\parindent}{0mm}

As with simulated annealing, the $successor()$ function adjusts $n$ of the six
parameters in current by a random amount, $\delta$, which may be sampled from
the same three distributions.

\section{Experiments and Results}

We begin by experimenting with variations on each of our learning algorithms. We
investigate the effects of differing types of successor functions. We also
experiment with modifications to our genetic algorithm by altering the initial
population size, the probability of mutation, and details of the reproduction
function. After determining the best set of variations for each learning
algorithm, we use our algorithms to search for the optimal parameters for
each commodity.

\subsection{Genetic Algorithm variations}


Our algorithm uses three pre-specified constants: the population size, $p$, the
number of parents used to reproduce, $n$, and the mutation probability,
$m$. $p$, specifies the total number of individuals at the beginning and end of
each iteration. During an iteration, reproduction doubles the population, and
then culling halves it. Our reproduction function takes $n$ parents, and adds
$n$ children to the population. After reproducing, the algorithm mutates each
new child, with probability $m$.

We investigate the effects of these constants on the overall learning rate of
the algorithm. We use the June crude oil commodity to experiment with specific
values for our pre-specified constants. We consider $p$ = 30, 60 and 120
individuals, $n$ = 2 and 6 parents, and $m$ = 0.1, 1 and 10 percent.  The
mutation function is fixed to alter one attribute at random.

During a single iteration, the $fitness()$ function is called once for each
individual.  Therefore, the total number of calls to the fitness function is the
population size multiplied by the number of iterations.  We fix the number of
calls to the fitness function to 1200 so that the experiments are
comparable. The number of iterations is reduced as we increase the $p$. For $p$
= 30, we run 40 iterations, for $p$ = 60, we run 20 iterations, and for $p$ =
120, we run 10 iterations.


Below, we present two plots of the profit earned versus the number of calls to
the $fitness()$ function. The first plot compares values of $m$ and $p$, for $n$
= 2; the second plot is for $n$ = 6.


PLOTS OF: PROFIT EARNED VS. MUTATION PROBABILITY, PROFIT EARNED VS. POPULATION
SIZE AND PROFIT EARNED VS. REPRODUCTION FUNCTION


\subsection{Successor Function}

A successor function takes in a node (i.e., a set of six parameters), and
modifies it in some way to produce a new node.  Our simulated annealing,
genetic, and random walk algorithms all use a successor function.  In the case
of our genetic algorithm, the successor function is the mutate function, which
takes in an individual, and mutates it to produce a slightly different
individual.

Our successor functions can be defined by two components, a distribution type,
and a number of parameters to modify, $n$. The parameters are modified by a
random amount $\delta$, which may be sampled from one of three distributions: a
uniformly random distribution, a normal distribution, or a constant distribution
($\delta$=1). We use the June crude oil commodity to experiment with all three
distributions for $n$ = 1, 3 and 6.

We conduct twenty-seven experiments. An experiment consists of a learning
algorithm, a distribution type, and a number of parameters to modify. For each
experiment, we train the learner, using the earliest two thirds of the June
crude oil data. In order to study the learning speed for each experiment, we
train the learner, using an increasing number of iterations, ranging from 0 to
1200.  At each step, we record the best parameters seen thus far.  We evaluate
the set of best parameters, using the testing data, and record the profit, for
each step. To minimize noise, we take an average of 100 trials at each
step. Below we present plots of the profit earned as a function of the number of
iterations, for each learning algorithm.


WE WILL INCLUDE THREE PLOTS OF PROFIT VS. NUMBER OF ITERATIONS, ONE FOR EACH
LEARNING ALGORITHM. EACH PLOT WILL HAVE 9 LINES, REPRESENTING 9 DIFFERENT
SUCCESSOR VARIATIONS FOR THAT LEARNING ALGORITHM.

\subsection{Finding Optimal Parameters}

After determining the successor functions and genetic algorithm variations, we
use each learning algorithm to find optimal parameters for each of the fourteen
commodities. For each commodity, we first train the learner over 300 iterations,
recording the most profitable set of six parameters. We then evaluate this set
of parameters, using the testing data.

Below we present the parameters obtained from each of our three learning
algorithms, along with Momsen's published parameter values, for all fourteen
commodities. For each set of parameters, we report the profit earned after
evaluating them with the testing data.


NOTE: THESE TABLES WILL BE FILLED IN FOR THE FINAL DRAFT.

\begin{tabular}{|r|l|l|l|l|l|l|l|}
  \hline
  April & \multicolumn{3}{|c|}{Channel Thresholds} & \multicolumn{3}{|c|}{Window Dates} &  \\
  Live Cattle & Entry & Trail Stop & Stop Loss & Open & Close & Exit & Profit\\ \hline
  Simulated Annealing & a & b & c & d & e & f & \$ \\ \hline
  Genetic Algorithms & a & b & c & d & e & f & \$ \\ \hline
  Random Walk & a & b & c & d & e & f & \$ \\ \hline
  Momsen & 25 & 22 & 2 & 12-02 & 02-01 & 03-23 & \$4,740 \\ \hline
\end{tabular}

\begin{tabular}{|r|l|l|l|l|l|l|l|}
  \hline
  August & \multicolumn{3}{|c|}{Channel Thresholds} & \multicolumn{3}{|c|}{Window Dates} &  \\
  Pork Bellies & Entry & Trail Stop & Stop Loss & Open & Close & Exit & Profit\\ \hline
  Simulated Annealing & a & b & c & d & e & f & \$ \\ \hline
  Genetic Algorithms & a & b & c & d & e & f & \$ \\ \hline
  Random Walk & a & b & c & d & e & f & \$ \\ \hline
  Momsen &  14 & 13 & 4 & 04-02 & 07-01 & 07-29 & \$30,030 \\ \hline
\end{tabular}

\begin{tabular}{|r|l|l|l|l|l|l|l|}
  \hline
  December & \multicolumn{3}{|c|}{Channel Thresholds} & \multicolumn{3}{|c|}{Window Dates} &  \\
  Corn & Entry & Trail Stop & Stop Loss & Open & Close & Exit & Profit\\ \hline
  Simulated Annealing & a & b & c & d & e & f & \$ \\ \hline
  Genetic Algorithms & a & b & c & d & e & f & \$ \\ \hline
  Random Walk & a & b & c & d & e & f & \$ \\ \hline
  Momsen & 21 & 11 & 2 & 05-12 & 08-01 & 08-06 & \$12,412 \\ \hline
\end{tabular}

\begin{tabular}{|r|l|l|l|l|l|l|l|}
  \hline
  December & \multicolumn{3}{|c|}{Channel Thresholds} & \multicolumn{3}{|c|}{Window Dates} &  \\
  Wheat & Entry & Trail Stop & Stop Loss & Open & Close & Exit & Profit\\ \hline
  Simulated Annealing & a & b & c & d & e & f & \$ \\ \hline
  Genetic Algorithms & a & b & c & d & e & f & \$ \\ \hline
  Random Walk & a & b & c & d & e & f & \$ \\ \hline
  Momsen & 12 & 5 & 3 & 05-20 & 08-01 & 08-06 & \$9,925 \\ \hline
\end{tabular}

\begin{tabular}{|r|l|l|l|l|l|l|l|}
  \hline
  December  & \multicolumn{3}{|c|}{Channel Thresholds} & \multicolumn{3}{|c|}{Window Dates} &  \\
  Lean Hogs & Entry & Trail Stop & Stop Loss & Open & Close & Exit & Profit\\ \hline
  Simulated Annealing & a & b & c & d & e & f & \$ \\ \hline
  Genetic Algorithms & a & b & c & d & e & f & \$ \\ \hline
  Random Walk & a & b & c & d & e & f & \$ \\ \hline
  Momsen & 12 & 12 & 5 & 05-15 & 07-01 & 08-12 & \$8,190 \\ \hline
\end{tabular}

\begin{tabular}{|r|l|l|l|l|l|l|l|}
  \hline
  February & \multicolumn{3}{|c|}{Channel Thresholds} & \multicolumn{3}{|c|}{Window Dates} &  \\
  Crude Oil & Entry & Trail Stop & Stop Loss & Open & Close & Exit & Profit\\ \hline
  Simulated Annealing & a & b & c & d & e & f & \$ \\ \hline
  Genetic Algorithms & a & b & c & d & e & f & \$ \\ \hline
  Random Walk & a & b & c & d & e & f & \$ \\ \hline
  Momsen & 12 & 15 & 4 & 10-11 & 12-01 & 12-19 & \$11,440 \\ \hline
\end{tabular}

\begin{tabular}{|r|l|l|l|l|l|l|l|}
  \hline
  June & \multicolumn{3}{|c|}{Channel Thresholds} & \multicolumn{3}{|c|}{Window Dates} &  \\
  Crude Oil & Entry & Trail Stop & Stop Loss & Open & Close & Exit & Profit\\ \hline
  Simulated Annealing & a & b & c & d & e & f & \$ \\ \hline
  Genetic Algorithms & a & b & c & d & e & f & \$ \\ \hline
  Random Walk & a & b & c & d & e & f & \$ \\ \hline
  Momsen & 17 & 14 & 3 & 02-24 & 04-01 & 04-20 & \$5,380 \\ \hline
\end{tabular}

\begin{tabular}{|r|l|l|l|l|l|l|l|}
  \hline
  June & \multicolumn{3}{|c|}{Channel Thresholds} & \multicolumn{3}{|c|}{Window Dates} &  \\
  Unleaded Gas & Entry & Trail Stop & Stop Loss & Open & Close & Exit & Profit\\ \hline
  Simulated Annealing & a & b & c & d & e & f & \$ \\ \hline
  Genetic Algorithms & a & b & c & d & e & f & \$ \\ \hline
  Random Walk & a & b & c & d & e & f & \$ \\ \hline
  Momsen & 6 & 8 & 2 & 03-02 & 04-13 & 05-09 & \$13,679 \\ \hline
\end{tabular}

\begin{tabular}{|r|l|l|l|l|l|l|l|}
  \hline
  June & \multicolumn{3}{|c|}{Channel Thresholds} & \multicolumn{3}{|c|}{Window Dates} &  \\
  Lean Hogs & Entry & Trail Stop & Stop Loss & Open & Close & Exit & Profit\\ \hline
  Simulated Annealing & a & b & c & d & e & f & \$ \\ \hline
  Genetic Algorithms & a & b & c & d & e & f & \$ \\ \hline
  Random Walk & a & b & c & d & e & f & \$ \\ \hline
  Momsen & 21 & 15 & 5 & 02-28 & 05-01 & 05-27 & \$4,260 \\ \hline
\end{tabular}

\begin{tabular}{|r|l|l|l|l|l|l|l|}
  \hline
  May & \multicolumn{3}{|c|}{Channel Thresholds} & \multicolumn{3}{|c|}{Window Dates} &  \\
  Heating Oil & Entry & Trail Stop & Stop Loss & Open & Close & Exit & Profit\\ \hline
  Simulated Annealing & a & b & c & d & e & f & \$ \\ \hline
  Genetic Algorithms & a & b & c & d & e & f & \$ \\ \hline
  Random Walk & a & b & c & d & e & f & \$ \\ \hline
  Momsen & 23 & 9 & 2 & 02-24 & 04-01 & 04-20 & \$2,431.80 \\ \hline
\end{tabular}

\begin{tabular}{|r|l|l|l|l|l|l|l|}
  \hline
  November & \multicolumn{3}{|c|}{Channel Thresholds} & \multicolumn{3}{|c|}{Window Dates} &  \\
  Crude Oil & Entry & Trail Stop & Stop Loss & Open & Close & Exit & Profit\\ \hline
  Simulated Annealing & a & b & c & d & e & f & \$ \\ \hline
  Genetic Algorithms & a & b & c & d & e & f & \$ \\ \hline
  Random Walk & a & b & c & d & e & f & \$ \\ \hline
  Momsen & 10 & 5 & 3 & 07-24 & 09-05 & 09-25 & \$9,150 \\ \hline
\end{tabular}

\begin{tabular}{|r|l|l|l|l|l|l|l|}
  \hline
  November & \multicolumn{3}{|c|}{Channel Thresholds} & \multicolumn{3}{|c|}{Window Dates} &  \\
  Heating Oil & Entry & Trail Stop & Stop Loss & Open & Close & Exit & Profit\\ \hline
  Simulated Annealing & a & b & c & d & e & f & \$ \\ \hline
  Genetic Algorithms & a & b & c & d & e & f & \$ \\ \hline
  Random Walk & a & b & c & d & e & f & \$ \\ \hline
  Momsen &  19 & 11 & 5 & 07-02 & 09-30 & 10-08 & \$13,981.80 \\ \hline
\end{tabular}

\begin{tabular}{|r|l|l|l|l|l|l|l|}
  \hline
  October & \multicolumn{3}{|c|}{Channel Thresholds} & \multicolumn{3}{|c|}{Window Dates} &  \\
  Lean Hogs & Entry & Trail Stop & Stop Loss & Open & Close & Exit & Profit\\ \hline
  Simulated Annealing & a & b & c & d & e & f & \$ \\ \hline
  Genetic Algorithms & a & b & c & d & e & f & \$ \\ \hline
  Random Walk & a & b & c & d & e & f & \$ \\ \hline
  Momsen & 15 & 7 & 1 & 08-29 & 09-30 & 10-01 & \$6,360 \\ \hline
\end{tabular}

\begin{tabular}{|r|l|l|l|l|l|l|l|}
  \hline
  September    & \multicolumn{3}{|c|}{Channel Thresholds} & \multicolumn{3}{|c|}{Window Dates} &  \\
  Orange Juice & Entry & Trail Stop & Stop Loss & Open & Close & Exit & Profit\\ \hline
  Simulated Annealing & a & b & c & d & e & f & \$ \\ \hline
  Genetic Algorithms & a & b & c & d & e & f & \$ \\ \hline
  Random Walk & a & b & c & d & e & f & \$ \\ \hline
  Momsen & 10 & 7 & 3 & 05-02 & 06-15 & 06-26 & \$1,890 \\ \hline
\end{tabular}

\section{Conclusions}

We intend to compare the learning rates for each of our three learning
algorithms. We hope to find that simulated annealing and our genetic algorithm
learn to increase the profit, more rapidly than our random walk algorithm. We
also compare the final parameters obtained for each of the learning algorithms
to Momsen's published parameters. We hope to find more profitable parameter's
than Momsen's.


\bibliographystyle{plain}
\bibliography{final}
\nocite{*}
\end{document}
\end

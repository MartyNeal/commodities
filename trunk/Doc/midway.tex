\documentclass[10pt]{article}
\usepackage{graphicx, url}
\usepackage[left=2cm,top=1cm,right=2.5cm,bottom=1cm,nohead,nofoot]{geometry}
\setlength{\parskip}{5mm plus5mm minus2mm}
\setlength{\parindent}{0mm}

\title{Machine learning optimal parameters for a channel breakout system for trading
commodities}
\author{Martin Neal\\
  Becky Engley}
\date{\today}

\begin{document}

\maketitle

\section{Introduction}

Commodities are raw materials and agricultural products that can be traded on
specialist commodities markets.  This paper aims to data mine commodity market
prices to enhance a predefined automated trading system.

The system to be enhanced is a channel breakout system described by John
Momsen's book, Superstar Seasonals.  A channel breakout system is a type of
trend following system which is a type of technical analysis.  Technical
analysis relies on the assumption that future market prices may be predicted by
studying past market prices and volume.  Trend following systems attempt to
determine when a market trend will begin and end, and make investments
accordingly.  A channel breakout system uses channel lines

Machine learning, a field closely related to data mining, is concerned with
algorithms which improve over time as additional data is obtained.  This paper
will attempt to independently use two machine learning techniques to generate
the best parameters possible for the channel breakout system.

%A commodity is either traded short or long.  If a trader expects an uptrend in
%the value of a commodity, he will trade long, buying before selling.  If a
%trader expects a downtrend, he will trade short, selling before buying.


\section{Related Work}

In his book, Superstar Seasonals, John Momson defines a channel breakout system
as a model for predicting uptrends and downtrends.  According to his system, a
commodity may be traded between a range of dates specified by the static parameters
$entry\_window\_open$ and $entry\_window\_close$.  A trade must be completed no
later than the $exit\_trade$ date, a third parameter of the system.

Momsen defines an upper channel line and a lower channel line.  The upper
channel line is determined by the maximum $high$ value over the previous $m$
days.  The lower channel line is determined by the minimum $low$ value over the
previous $n$ days.  Here, $m$ and $n$ are also parameters of the system and are
different for each commodity.  If trading long, the upper channel line is the
$entry$ line and the lower channel line is the $trail$-$stop$ line.  If trading
short, the channel lines are reversed.

The sixth and final system parameter, the $stop$-$loss$ threshold, serves as a
safety net to prevent large losses.  When trading short, the stop-loss is
defined by the maximum $high$ value over the previous $q$ days, where $q$ is
strictly less than $m$.  When trading long, the stop-loss is defined by the
minimum $low$ value over the previous $q$ days, where $q$ is strictly less than
$n$.  The trail-stop line protects profits, while the stop-loss threshold
minimizes losses.

According to the system, if the current value of the commodity crosses the entry
line, it indicates the beginning of a new trend and the trade is begun.  When
the current value of the commodity crosses the trail-stop line, the trade is
completed.  More often than not, the cross of a channel line incorrectly
predicts the beginning of a trend.  In these cases, small losses are incurred.
However, when the system correctly predicts a long trend, the profits generated
by this trend more than make up for small losses.

Over the past thirty years, Momsen has traded this system, with outstanding
results.  We intend to extend his work, by machine learning optimal values for
the six trading parameters.

\section{Methods}

Simulated annealing and genetic algorithms are well suited for optimization
problems such as these because they can find satisfactory solutions in large
hypothesis spaces, and use a constant amount of memory by saving only the
currently best solution to the problem and the algorithms proceed.


%introduce methods and algorithms here

\subsection{Data Set}

We have financial market data for eighteen commodities over a period of thirty
years.  Because the learning system is evaluated per contract year, each year
comprises a single data point.  We have approximately thirty data points for
each commodity.  The commodities include live cattle, pork bellies, corn, wheat,
lean hogs, crude oil, sugar, unleaded gasoline, heating oil, soybeans, cotton,
orange juice, and soybean oil.  Some commodities are traded during multiple
seasons.  The data includes open, high, low, and closing values for each
commodity on every trading day.

The data set will be partitioned into two sets, training data and testing data.
For each commodity, we will use the only two thirds of the data to train the
system, reserving the remaining data to test the system.  We will use the most
recent data for testing.  No validation data is reserved because our model is
predetermined.

\subsection{Algorithms}

Optimal parameter values maximize the profit earned as a result of trading the
system.  We will approach this optimization problem using two different machine
learning techniques: simulated annealing, and genetic algorithms.  Below we
present these two algorithms, along with the algorithm for our channel breakout
system.

\subsubsection{The Channel Breakout System}

The algorithm for the channel breakout system takes in the six system trading
parameters, and returns the total profit earned. Our tradeSystem function
represents our objective function, the return value of which we are trying to
maximize.  It is used as the $value()$ function for the Simulated Annealing algorithm,
and the fitness function for the Genetic algorithm, both defined below.

\setlength{\parindent}{5mm}
\indent tradeSystem($entry\_window\_open$, $entry\_window\_close$, $exit\_date$,\\
\indent \indent \indent \indent \indent $entry\_threshold$, $trailstop\_threshold$, $stoploss\_threshold$)\\
\indent \indent compute entry, trailstop and stoploss channels, using thresholds\\
\indent \indent do\\
\indent \indent \indent if(in\_trade)\\
\indent \indent \indent \indent if(entry line is crossed)\\
\indent \indent \indent \indent \indent in\_trade $\leftarrow$ TRUE\\
\indent \indent \indent \indent \indent entry\_price $\leftarrow$ current\_price\\
\indent \indent \indent \indent \indent stoploss value $\leftarrow$ current\_price + stoploss\_threshold\\
\indent \indent \indent else\\
\indent \indent \indent \indent if(trailstop\_ threshold $>$ current\_price or\\
\indent \indent \indent \indent \indent stoploss\_threshold $>$ current\_price or\\
\indent \indent \indent \indent \indent exit\_date = current\_date)\\
\indent \indent \indent \indent \indent in\_trade $\leftarrow$ FALSE\\
\indent \indent \indent \indent \indent exit\_price $\leftarrow$ current\_price\\
\indent \indent while(current\_date $<$ entry\_window\_close or in a trade)\\
\indent \indent return profit\\
\setlength{\parindent}{0mm}

We may repeatedly enter and then exit a trade many times over the course of one
trading year. When in a trade, we check every day, to determine if we should
exit; when not currently in a trade, we check to determine if we should enter,
until the entry window closes. When we enter, we compute the entry price based
on the value at which we crossed the entry channel line. The stoploss value is
also based on this cross-point. When we exit, we compute the exit price based on
the value at which we crossed the closer of the trail stop channel and the stop
loss channel. The profit for this trade is the difference between the entry and
exit prices.

\subsubsection{Simulated Annealing}
Simulated Annealing combines the best of hill climbing and random walk heuristic
algorithms.  The major problem with hill climbing algorithms is that they can
get stuck on local maxima, because they never move downhill.
Random walks, however, are guaranteed to find the global maximum but take far
too long to do so.  Simulated Annealing combines these approaches, yielding both
efficiency and completeness.  We present the psuedo-code for the Simulated
Annealing algorithm below.

\setlength{\parindent}{5mm}
\indent simulated-annealing()\\
\indent \indent current $\leftarrow$ 6 random parameter values\\
\indent \indent for t $\leftarrow$ 1 to $\infty$\\
\indent \indent \indent $\eta \leftarrow f$(t)\\
\indent \indent \indent next $\leftarrow$ successor(current)\\
\indent \indent \indent $\Delta E \leftarrow$ value(next) - value(current)\\
\indent \indent \indent if($\Delta E > 0$)\\
\indent \indent \indent \indent current $\leftarrow$ next\\
\indent \indent \indent else\\
\indent \indent \indent \indent current $\leftarrow$ next only with probability $e^{\Delta E\eta}$\\
\indent \indent return current\\
\setlength{\parindent}{0mm}

The value function trades the system with the six parameters and returns the
profit made (or lost).  current and next are both nodes.  In this context, a
node is a set of values for the six parameters.  $\eta$ controls the probability
of downward steps.  $f$(t) is a decreasing function of time; it decreases the
probability of downward steps as time increases.  This may be done in any number
of ways (e.g. linearly, exponentially).  The choice for the successor function
greatly affects the learning speed of the algorithm.  Our successor function
will adjust exactly one of the six parameters in current by a random amount,
$\delta$, which may be either uniformly or normally distributed.  As a variation
on this function, we can change multiple parameters at once.  We intend to
experiment with all of these variations.
\subsubsection{Genetic Algorithm}
Genetic Algorithms model evolutionary processes.  In our proposed
implementation, a set of six system parameters represents an individual.  Many
individuals form a population which is repeatedly bred and then culled.
Breeding swaps random parameters from two or more parents to create children.
Culling evaluates each individual using a fitness function, and eliminates unfit
individuals from the population.  The fitness function returns the profit earned
by trading the channel breakout system using the individuals' six parameters.
We present the psuedo-code for the Genetic Algorithm below.

\setlength{\parindent}{5mm}
\indent genetic-algorithm(population)\\
\indent \indent do\\
\indent \indent \indent for i $\leftarrow$ 1 to size(population)\\
\indent \indent \indent \indent parents $\leftarrow$ random-subset(population,size)\\
\indent \indent \indent \indent children $\leftarrow$ reproduce(parents)\\
\indent \indent \indent \indent children $\leftarrow$ mutate(children) with small random probability\\
\indent \indent \indent \indent population.add(children)\\
\indent \indent \indent population $\leftarrow$ cull(population,threshold)\\
\indent \indent until (enough time has elapsed)\\
\indent \indent return best-individual(population)\\
\setlength{\parindent}{0mm}

This generic algorithm leaves a lot to be decided by the implementer.
The initial population size may vary.  The mutate probability may be changed.
The mutation of an attribute may be uniformly random or normally distributed.
The reproduce function may take between two and six parents.  These parents
randomly swap attributes to produce one or more children, which are added to the
population.  The population is then culled, which preserves a threshold
number of individuals, removing all others as unfit.

\subsubsection{A Random Learner}

Here we describe a simple learner for the maximizing the parameters to
$tradeSystem()$.  This learner begins by randomly generating parameters as its
best guess for maximizing profits.  It then loops for a set number of iterations,
randomly choosing  six parameters at a time, keeping the best set of parameters.

\setlength{\parindent}{5mm}
\indent random-algorithm(epochs)\\
\indent \indent current $\leftarrow$ 6 random parameter values\\
\indent \indent for t $\leftarrow$ 1 to epochs\\
\indent \indent \indent next $\leftarrow$ 6 random parameter values\\
\indent \indent \indent if(value(next) > value(current))\\
\indent \indent \indent \indent current $\leftarrow$ next\\
\indent \indent return current\\
\setlength{\parindent}{0mm}

\section{Experiments and Results}

We were interested in two attributes of our learning system.  We wanted
to know how well each learner did given a lot of time, as well as how quickly it
learned.  We discuss each statistic below.

\section{Profits}

Below is the are the parameters obtained for each commodities using three
systems, simulated annealing, genetic algorithm, and Momsens's parameters.  Also
indicated is the profits obtained using the parameters.

NOTE: THESE TABLES WILL BE FILLED IN FOR THE FINAL DRAFT.

\begin{tabular}{|r|l|l|l|l|l|l|l|}
  \hline
  April & \multicolumn{3}{|c|}{Channel Sizes} & \multicolumn{3}{|c|}{Dates} &  \\
  Live Cattle & Entry & Trail Stop & Stop Loss & Entry & No Entry & Exit & Profit\\ \hline
  Simulated Annealing & a & b & c & d & e & f & \$ \\ \hline
  Genetic Algorithms & a & b & c & d & e & f & \$ \\ \hline
  Momsen &  a & b & c & d & e & f & \$ \\ \hline
\end{tabular}

\begin{tabular}{|r|l|l|l|l|l|l|l|}
  \hline
  August & \multicolumn{3}{|c|}{Channel Sizes} & \multicolumn{3}{|c|}{Dates} &  \\
  Pork Bellies & Entry & Trail Stop & Stop Loss & Entry & No Entry & Exit & Profit\\ \hline
  Simulated Annealing & a & b & c & d & e & f & \$ \\ \hline
  Genetic Algorithms & a & b & c & d & e & f & \$ \\ \hline
  Momsen &  a & b & c & d & e & f & \$ \\ \hline
\end{tabular}

\begin{tabular}{|r|l|l|l|l|l|l|l|}
  \hline
  December & \multicolumn{3}{|c|}{Channel Sizes} & \multicolumn{3}{|c|}{Dates} &  \\
  Corn & Entry & Trail Stop & Stop Loss & Entry & No Entry & Exit & Profit\\ \hline
  Simulated Annealing & a & b & c & d & e & f & \$ \\ \hline
  Genetic Algorithms & a & b & c & d & e & f & \$ \\ \hline
  Momsen &  a & b & c & d & e & f & \$ \\ \hline
\end{tabular}

\begin{tabular}{|r|l|l|l|l|l|l|l|}
  \hline
  December & \multicolumn{3}{|c|}{Channel Sizes} & \multicolumn{3}{|c|}{Dates} &  \\
  Wheat & Entry & Trail Stop & Stop Loss & Entry & No Entry & Exit & Profit\\ \hline
  Simulated Annealing & a & b & c & d & e & f & \$ \\ \hline
  Genetic Algorithms & a & b & c & d & e & f & \$ \\ \hline
  Momsen &  a & b & c & d & e & f & \$ \\ \hline
\end{tabular}

\begin{tabular}{|r|l|l|l|l|l|l|l|}
  \hline
  December  & \multicolumn{3}{|c|}{Channel Sizes} & \multicolumn{3}{|c|}{Dates} &  \\
  Lean Hogs & Entry & Trail Stop & Stop Loss & Entry & No Entry & Exit & Profit\\ \hline
  Simulated Annealing & a & b & c & d & e & f & \$ \\ \hline
  Genetic Algorithms & a & b & c & d & e & f & \$ \\ \hline
  Momsen &  a & b & c & d & e & f & \$ \\ \hline
\end{tabular}

\begin{tabular}{|r|l|l|l|l|l|l|l|}
  \hline
  February & \multicolumn{3}{|c|}{Channel Sizes} & \multicolumn{3}{|c|}{Dates} &  \\
  Crude Oil & Entry & Trail Stop & Stop Loss & Entry & No Entry & Exit & Profit\\ \hline
  Simulated Annealing & a & b & c & d & e & f & \$ \\ \hline
  Genetic Algorithms & a & b & c & d & e & f & \$ \\ \hline
  Momsen &  a & b & c & d & e & f & \$ \\ \hline
\end{tabular}

\begin{tabular}{|r|l|l|l|l|l|l|l|}
  \hline
  July & \multicolumn{3}{|c|}{Channel Sizes} & \multicolumn{3}{|c|}{Dates} &  \\
  Soy Beans & Entry & Trail Stop & Stop Loss & Entry & No Entry & Exit & Profit\\ \hline
  Simulated Annealing & a & b & c & d & e & f & \$ \\ \hline
  Genetic Algorithms & a & b & c & d & e & f & \$ \\ \hline
  Momsen &  a & b & c & d & e & f & \$ \\ \hline
\end{tabular}

\begin{tabular}{|r|l|l|l|l|l|l|l|}
  \hline
  June & \multicolumn{3}{|c|}{Channel Sizes} & \multicolumn{3}{|c|}{Dates} &  \\
  Crude Oil & Entry & Trail Stop & Stop Loss & Entry & No Entry & Exit & Profit\\ \hline
  Simulated Annealing & a & b & c & d & e & f & \$ \\ \hline
  Genetic Algorithms & a & b & c & d & e & f & \$ \\ \hline
  Momsen &  a & b & c & d & e & f & \$ \\ \hline
\end{tabular}

\begin{tabular}{|r|l|l|l|l|l|l|l|}
  \hline
  June & \multicolumn{3}{|c|}{Channel Sizes} & \multicolumn{3}{|c|}{Dates} &  \\
  Unleaded Gas & Entry & Trail Stop & Stop Loss & Entry & No Entry & Exit & Profit\\ \hline
  Simulated Annealing & a & b & c & d & e & f & \$ \\ \hline
  Genetic Algorithms & a & b & c & d & e & f & \$ \\ \hline
  Momsen &  a & b & c & d & e & f & \$ \\ \hline
\end{tabular}

\begin{tabular}{|r|l|l|l|l|l|l|l|}
  \hline
  June & \multicolumn{3}{|c|}{Channel Sizes} & \multicolumn{3}{|c|}{Dates} &  \\
  Lean Hogs & Entry & Trail Stop & Stop Loss & Entry & No Entry & Exit & Profit\\ \hline
  Simulated Annealing & a & b & c & d & e & f & \$ \\ \hline
  Genetic Algorithms & a & b & c & d & e & f & \$ \\ \hline
  Momsen &  a & b & c & d & e & f & \$ \\ \hline
\end{tabular}

\begin{tabular}{|r|l|l|l|l|l|l|l|}
  \hline
  May & \multicolumn{3}{|c|}{Channel Sizes} & \multicolumn{3}{|c|}{Dates} &  \\
  Heating Oil & Entry & Trail Stop & Stop Loss & Entry & No Entry & Exit & Profit\\ \hline
  Simulated Annealing & a & b & c & d & e & f & \$ \\ \hline
  Genetic Algorithms & a & b & c & d & e & f & \$ \\ \hline
  Momsen &  a & b & c & d & e & f & \$ \\ \hline
\end{tabular}

\begin{tabular}{|r|l|l|l|l|l|l|l|}
  \hline
  November & \multicolumn{3}{|c|}{Channel Sizes} & \multicolumn{3}{|c|}{Dates} &  \\
  Crude Oil & Entry & Trail Stop & Stop Loss & Entry & No Entry & Exit & Profit\\ \hline
  Simulated Annealing & a & b & c & d & e & f & \$ \\ \hline
  Genetic Algorithms & a & b & c & d & e & f & \$ \\ \hline
  Momsen &  a & b & c & d & e & f & \$ \\ \hline
\end{tabular}

\begin{tabular}{|r|l|l|l|l|l|l|l|}
  \hline
  October & \multicolumn{3}{|c|}{Channel Sizes} & \multicolumn{3}{|c|}{Dates} &  \\
  Lean Hogs & Entry & Trail Stop & Stop Loss & Entry & No Entry & Exit & Profit\\ \hline
  Simulated Annealing & a & b & c & d & e & f & \$ \\ \hline
  Genetic Algorithms & a & b & c & d & e & f & \$ \\ \hline
  Momsen &  a & b & c & d & e & f & \$ \\ \hline
\end{tabular}

\begin{tabular}{|r|l|l|l|l|l|l|l|}
  \hline
  September    & \multicolumn{3}{|c|}{Channel Sizes} & \multicolumn{3}{|c|}{Dates} &  \\
  Orange Juice & Entry & Trail Stop & Stop Loss & Entry & No Entry & Exit & Profit\\ \hline
  Simulated Annealing & a & b & c & d & e & f & \$ \\ \hline
  Genetic Algorithms &p a & b & c & d & e & f & \$ \\ \hline
  Momsen &  a & b & c & d & e & f & \$ \\ \hline
\end{tabular}

\subsection{Learning Speed}

Here is a chart of the average profit over all commodities as a function of
time.  Each unit of time represents an epoch or an iteration through the main
loop of the algorithm during training over the training data.  The y-axis is the
total profit earned from trading the system using the parameters on the testing
data.

PLOT AND DISCUSSION OF PLOT GOES HERE.\\
COMPARISON OF SIMULATED ANNEALING, GENETIC ALGORITHM, AND RANDOM ALGORITHM.\\

\section{Conclusions}



%\bibliographystyle{plain}
%\bibliography{571writeup}
%\nocite{*}
\end{document}
